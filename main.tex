\documentclass{paper}

\usepackage{fontspec}
\usepackage{geometry}
\usepackage{listings}

\usepackage[unicode,pdfencoding=auto]{hyperref}
\usepackage[hyperref, UTF8]{ctex}

\geometry{left=2.5cm, right=2.5cm, top=3.6cm, bottom=3.6cm}
\setmainfont{Monaco}

\hypersetup{colorlinks=true,linkcolor=black}


\begin{document}

\begin{titlepage}
	\centering
	\vfill
	{\bfseries\Huge
		Golang 最佳实践\\
		\vskip1cm
		\Large
		袁胜 \\
	}
	\vfill
	\includegraphics[width=8cm]{golang.png} % also works with logo.pdf
	\vfill
	\vfill
\end{titlepage}

\tableofcontents

\newpage

\section{指导原则}

\subsection{简单性}

\subsection{可读性}

\subsection{生产力}


\section{标识符}

\subsection{选择标识是为了清晰, 而不是简洁}

\subsection{标识符长度}

\subsection{不要用变量类型命名变量}

\subsection{使用一致的命名风格}

\subsection{使用一致的声明样式}

\subsection{成为团队的合作者}

\section{注释}

\subsection{关于变量和常量的注释应描述其内容而非其目的 }
\subsection{公共符号始终要注释}

\section{包的设计}

\subsection{一个好的包从它的名字开始}
\subsection{避免使用类似 base、 common或 util的包名称}
\subsection{尽早 return而不是深度嵌套}
\subsection{让零值更有用}
\subsection{避免包级别状态}

\section{项目结构}

\subsection{考虑更少,更大的包}
\subsection{保持 main包内容尽可能的少}

\section{API 设计}

\subsection{设计难以被误用的 API}
\subsection{为其默认用例设计 API}
\subsection{让函数定义它们所需的行为}

\section{错误处理}

\subsection{通过消除错误来消除错误处理}
\subsection{错误只处理一次}

\section{并发}
\subsection{保持自己忙碌或做自己的工作}
\subsection{将并发性留给调用者}
\subsection{永远不要启动一个停止不了的 goroutine}

\end{document}